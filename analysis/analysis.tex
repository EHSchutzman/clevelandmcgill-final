\documentclass[]{article}
\usepackage{lmodern}
\usepackage{amssymb,amsmath}
\usepackage{ifxetex,ifluatex}
\ifnum 0\ifxetex 1\fi\ifluatex 1\fi=0 % if pdftex
  \usepackage[T1]{fontenc}
  \usepackage[utf8]{inputenc}
\else % if luatex or xelatex
  \ifxetex
    \usepackage{mathspec}
  \else
    \usepackage{fontspec}
  \fi
  \defaultfontfeatures{Ligatures=TeX,Scale=MatchLowercase}
\fi
% use upquote if available, for straight quotes in verbatim environments
\IfFileExists{upquote.sty}{\usepackage{upquote}}{}
% use microtype if available
\IfFileExists{microtype.sty}{%
\usepackage{microtype}
\UseMicrotypeSet[protrusion]{basicmath} % disable protrusion for tt fonts
}{}
\usepackage[margin=1in]{geometry}
\usepackage{hyperref}
\hypersetup{unicode=true,
            pdftitle={The Effect of Time},
            pdfauthor={Ethan Schutzman},
            pdfborder={0 0 0},
            breaklinks=true}
\urlstyle{same}  % don't use monospace font for urls
\usepackage{color}
\usepackage{fancyvrb}
\newcommand{\VerbBar}{|}
\newcommand{\VERB}{\Verb[commandchars=\\\{\}]}
\DefineVerbatimEnvironment{Highlighting}{Verbatim}{commandchars=\\\{\}}
% Add ',fontsize=\small' for more characters per line
\usepackage{framed}
\definecolor{shadecolor}{RGB}{248,248,248}
\newenvironment{Shaded}{\begin{snugshade}}{\end{snugshade}}
\newcommand{\KeywordTok}[1]{\textcolor[rgb]{0.13,0.29,0.53}{\textbf{{#1}}}}
\newcommand{\DataTypeTok}[1]{\textcolor[rgb]{0.13,0.29,0.53}{{#1}}}
\newcommand{\DecValTok}[1]{\textcolor[rgb]{0.00,0.00,0.81}{{#1}}}
\newcommand{\BaseNTok}[1]{\textcolor[rgb]{0.00,0.00,0.81}{{#1}}}
\newcommand{\FloatTok}[1]{\textcolor[rgb]{0.00,0.00,0.81}{{#1}}}
\newcommand{\ConstantTok}[1]{\textcolor[rgb]{0.00,0.00,0.00}{{#1}}}
\newcommand{\CharTok}[1]{\textcolor[rgb]{0.31,0.60,0.02}{{#1}}}
\newcommand{\SpecialCharTok}[1]{\textcolor[rgb]{0.00,0.00,0.00}{{#1}}}
\newcommand{\StringTok}[1]{\textcolor[rgb]{0.31,0.60,0.02}{{#1}}}
\newcommand{\VerbatimStringTok}[1]{\textcolor[rgb]{0.31,0.60,0.02}{{#1}}}
\newcommand{\SpecialStringTok}[1]{\textcolor[rgb]{0.31,0.60,0.02}{{#1}}}
\newcommand{\ImportTok}[1]{{#1}}
\newcommand{\CommentTok}[1]{\textcolor[rgb]{0.56,0.35,0.01}{\textit{{#1}}}}
\newcommand{\DocumentationTok}[1]{\textcolor[rgb]{0.56,0.35,0.01}{\textbf{\textit{{#1}}}}}
\newcommand{\AnnotationTok}[1]{\textcolor[rgb]{0.56,0.35,0.01}{\textbf{\textit{{#1}}}}}
\newcommand{\CommentVarTok}[1]{\textcolor[rgb]{0.56,0.35,0.01}{\textbf{\textit{{#1}}}}}
\newcommand{\OtherTok}[1]{\textcolor[rgb]{0.56,0.35,0.01}{{#1}}}
\newcommand{\FunctionTok}[1]{\textcolor[rgb]{0.00,0.00,0.00}{{#1}}}
\newcommand{\VariableTok}[1]{\textcolor[rgb]{0.00,0.00,0.00}{{#1}}}
\newcommand{\ControlFlowTok}[1]{\textcolor[rgb]{0.13,0.29,0.53}{\textbf{{#1}}}}
\newcommand{\OperatorTok}[1]{\textcolor[rgb]{0.81,0.36,0.00}{\textbf{{#1}}}}
\newcommand{\BuiltInTok}[1]{{#1}}
\newcommand{\ExtensionTok}[1]{{#1}}
\newcommand{\PreprocessorTok}[1]{\textcolor[rgb]{0.56,0.35,0.01}{\textit{{#1}}}}
\newcommand{\AttributeTok}[1]{\textcolor[rgb]{0.77,0.63,0.00}{{#1}}}
\newcommand{\RegionMarkerTok}[1]{{#1}}
\newcommand{\InformationTok}[1]{\textcolor[rgb]{0.56,0.35,0.01}{\textbf{\textit{{#1}}}}}
\newcommand{\WarningTok}[1]{\textcolor[rgb]{0.56,0.35,0.01}{\textbf{\textit{{#1}}}}}
\newcommand{\AlertTok}[1]{\textcolor[rgb]{0.94,0.16,0.16}{{#1}}}
\newcommand{\ErrorTok}[1]{\textcolor[rgb]{0.64,0.00,0.00}{\textbf{{#1}}}}
\newcommand{\NormalTok}[1]{{#1}}
\usepackage{graphicx}
% grffile has become a legacy package: https://ctan.org/pkg/grffile
\IfFileExists{grffile.sty}{%
\usepackage{grffile}
}{}
\makeatletter
\def\maxwidth{\ifdim\Gin@nat@width>\linewidth\linewidth\else\Gin@nat@width\fi}
\def\maxheight{\ifdim\Gin@nat@height>\textheight\textheight\else\Gin@nat@height\fi}
\makeatother
% Scale images if necessary, so that they will not overflow the page
% margins by default, and it is still possible to overwrite the defaults
% using explicit options in \includegraphics[width, height, ...]{}
\setkeys{Gin}{width=\maxwidth,height=\maxheight,keepaspectratio}
\IfFileExists{parskip.sty}{%
\usepackage{parskip}
}{% else
\setlength{\parindent}{0pt}
\setlength{\parskip}{6pt plus 2pt minus 1pt}
}
\setlength{\emergencystretch}{3em}  % prevent overfull lines
\providecommand{\tightlist}{%
  \setlength{\itemsep}{0pt}\setlength{\parskip}{0pt}}
\setcounter{secnumdepth}{0}
% Redefines (sub)paragraphs to behave more like sections
\ifx\paragraph\undefined\else
\let\oldparagraph\paragraph
\renewcommand{\paragraph}[1]{\oldparagraph{#1}\mbox{}}
\fi
\ifx\subparagraph\undefined\else
\let\oldsubparagraph\subparagraph
\renewcommand{\subparagraph}[1]{\oldsubparagraph{#1}\mbox{}}
\fi

%%% Use protect on footnotes to avoid problems with footnotes in titles
\let\rmarkdownfootnote\footnote%
\def\footnote{\protect\rmarkdownfootnote}

%%% Change title format to be more compact
\usepackage{titling}

% Create subtitle command for use in maketitle
\providecommand{\subtitle}[1]{
  \posttitle{
    \begin{center}\large#1\end{center}
    }
}

\setlength{\droptitle}{-2em}

  \title{The Effect of Time}
    \pretitle{\vspace{\droptitle}\centering\huge}
  \posttitle{\par}
    \author{Ethan Schutzman}
    \preauthor{\centering\large\emph}
  \postauthor{\par}
      \predate{\centering\large\emph}
  \postdate{\par}
    \date{December 09, 2019}


\begin{document}
\maketitle

Analysis of data collected from a Cleveland \& McGill style study
investigating the effects of time.

References:\\
- \url{https://github.com/mjskay/tidybayes}\\
- @codementum

\subsection{Libraries needed}\label{libraries-needed}

\begin{Shaded}
\begin{Highlighting}[]
\KeywordTok{library}\NormalTok{(}\StringTok{"jsonlite"}\NormalTok{)}
\KeywordTok{library}\NormalTok{(RCurl)}
\KeywordTok{library}\NormalTok{(plyr)}
\KeywordTok{library}\NormalTok{(tidyverse)}
\KeywordTok{library}\NormalTok{(uuid)}
\end{Highlighting}
\end{Shaded}

\subsection{Grab the JSON file from firebase and convert into a list in
R}\label{grab-the-json-file-from-firebase-and-convert-into-a-list-in-r}

\begin{Shaded}
\begin{Highlighting}[]
\NormalTok{tables <-}\StringTok{ }\KeywordTok{fromJSON}\NormalTok{(}\StringTok{"https://clevelandmcgill-final.firebaseio.com/.json"}\NormalTok{)}
\end{Highlighting}
\end{Shaded}

\subsection{\texorpdfstring{Create the \texttt{sessions}
tibble}{Create the sessions tibble}}\label{create-the-sessions-tibble}

\begin{Shaded}
\begin{Highlighting}[]
\NormalTok{sessions <-}\StringTok{ }\KeywordTok{tibble}\NormalTok{()}
\NormalTok{for (session in tables$Session) \{}
  \NormalTok{new_row <-}\StringTok{ }\KeywordTok{as_tibble}\NormalTok{(session)}
  \NormalTok{sessions <-}\StringTok{ }\NormalTok{sessions %>%}\StringTok{ }\KeywordTok{bind_rows}\NormalTok{(new_row)}
\NormalTok{\}}
\end{Highlighting}
\end{Shaded}

\subsection{\texorpdfstring{Create the \texttt{trials}
tibble}{Create the trials tibble}}\label{create-the-trials-tibble}

\begin{Shaded}
\begin{Highlighting}[]
\CommentTok{#tables$Trial}
\NormalTok{trials =}\StringTok{ }\KeywordTok{tibble}\NormalTok{()}
\NormalTok{for (img_name in tables$Trial) \{}
  \CommentTok{#img_name}
  \NormalTok{for (observation in img_name) \{}
      \NormalTok{new_row <-}\StringTok{ }\KeywordTok{as_tibble}\NormalTok{(observation)}
      \NormalTok{trials <-}\StringTok{ }\NormalTok{trials %>%}\StringTok{ }\KeywordTok{bind_rows}\NormalTok{(new_row)}
    \CommentTok{# trials <- rbind(trials_df, data.frame(observation))}
  \NormalTok{\}}
\NormalTok{\}}
\CommentTok{#Filter incomplete trials to not skew data.}
 \NormalTok{trials <-}\StringTok{ }\NormalTok{trials %>%}
\StringTok{  }\KeywordTok{filter}\NormalTok{(session_id !=}\StringTok{ "c071ee29-5210-498b-9af4-3f361a9a68ba"} \NormalTok{&}\StringTok{ }\NormalTok{session_id !=}\StringTok{ "98f79c48-4803-4300-8c92-8c51d5c0c20c"} \NormalTok{&}\StringTok{ }\NormalTok{session_id !=}\StringTok{ "1a98794c-1fff-40d4-9ddf-ff42f8218906"} \NormalTok{&}\StringTok{ }\NormalTok{session_id !=}\StringTok{ "174ac2ef-bc7e-4f08-9185-fb1c99eb6ffc"}\NormalTok{)}
\end{Highlighting}
\end{Shaded}

\subsection{Clean the data to make later function calls
easier}\label{clean-the-data-to-make-later-function-calls-easier}

\subsubsection{\texorpdfstring{Create \texttt{condition} column from
\texttt{image\_name} (to easily filter by
condition)}{Create condition column from image\_name (to easily filter by condition)}}\label{create-condition-column-from-image_name-to-easily-filter-by-condition}

\begin{Shaded}
\begin{Highlighting}[]
\NormalTok{trials$condition <-}\StringTok{ }\OtherTok{NA}
\NormalTok{for (i in }\KeywordTok{seq_along}\NormalTok{(trials$image_name)) \{}
  \NormalTok{if (}\KeywordTok{length}\NormalTok{(}\KeywordTok{grep}\NormalTok{(}\StringTok{"B"}\NormalTok{, trials$image_name[i])) >}\StringTok{ }\DecValTok{0}\NormalTok{)}
    \NormalTok{condition <-}\StringTok{ 'Bar'}
  \NormalTok{else if (}\KeywordTok{length}\NormalTok{(}\KeywordTok{grep}\NormalTok{(}\StringTok{"P"}\NormalTok{, trials$image_name[i])) >}\StringTok{ }\DecValTok{0}\NormalTok{)}
    \NormalTok{condition <-}\StringTok{ 'Pie'}
  \NormalTok{else}
    \NormalTok{condition <-}\StringTok{ '???'}
  
  \NormalTok{trials$condition[i] <-}\StringTok{ }\NormalTok{condition}
\NormalTok{\}}
\end{Highlighting}
\end{Shaded}

\begin{Shaded}
\begin{Highlighting}[]
\NormalTok{trials$time <-}\StringTok{ }\OtherTok{NA}
\NormalTok{for (i in }\KeywordTok{seq_along}\NormalTok{(trials$image_name)) \{}
  \NormalTok{if (}\KeywordTok{length}\NormalTok{(}\KeywordTok{grep}\NormalTok{(}\StringTok{"1000"}\NormalTok{, trials$image_name[i])) >}\StringTok{ }\DecValTok{0}\NormalTok{)}
    \NormalTok{time <-}\StringTok{ 'One Second'}
  \NormalTok{else if (}\KeywordTok{length}\NormalTok{(}\KeywordTok{grep}\NormalTok{(}\StringTok{"0500"}\NormalTok{, trials$image_name[i])) >}\StringTok{ }\DecValTok{0}\NormalTok{)}
    \NormalTok{time <-}\StringTok{ 'Half Second'}
  \NormalTok{else if (}\KeywordTok{length}\NormalTok{(}\KeywordTok{grep}\NormalTok{(}\StringTok{"3000"}\NormalTok{, trials$image_name[i])) >}\StringTok{ }\DecValTok{0}\NormalTok{)}
    \NormalTok{time <-}\StringTok{ 'Three Seconds'}
  \NormalTok{else}
    \NormalTok{time <-}\StringTok{ "???"}
  
  \NormalTok{trials$time[i] <-}\StringTok{ }\NormalTok{time}
\NormalTok{\}}
\end{Highlighting}
\end{Shaded}

\begin{Shaded}
\begin{Highlighting}[]
\NormalTok{trials$time_chart <-}\StringTok{ }\OtherTok{NA}
\NormalTok{for (i in }\KeywordTok{seq_along}\NormalTok{(trials$image_name)) \{}
  \NormalTok{if(}\KeywordTok{length}\NormalTok{(}\KeywordTok{grep}\NormalTok{(}\StringTok{"B"}\NormalTok{, trials$image_name[i]) >}\StringTok{ }\DecValTok{0}\NormalTok{))\{}
    \NormalTok{if (}\KeywordTok{length}\NormalTok{(}\KeywordTok{grep}\NormalTok{(}\StringTok{"1000"}\NormalTok{, trials$image_name[i])) >}\StringTok{ }\DecValTok{0}\NormalTok{)}
      \NormalTok{time_chart <-}\StringTok{ 'One Second Bar'}
    \NormalTok{else if (}\KeywordTok{length}\NormalTok{(}\KeywordTok{grep}\NormalTok{(}\StringTok{"0500"}\NormalTok{, trials$image_name[i])) >}\StringTok{ }\DecValTok{0}\NormalTok{)}
      \NormalTok{time_chart <-}\StringTok{ 'Half Second Bar'}
    \NormalTok{else if (}\KeywordTok{length}\NormalTok{(}\KeywordTok{grep}\NormalTok{(}\StringTok{"3000"}\NormalTok{, trials$image_name[i])) >}\StringTok{ }\DecValTok{0}\NormalTok{)}
      \NormalTok{time_chart <-}\StringTok{ 'Three Seconds Bar'}
    \NormalTok{else}
      \NormalTok{time_chart <-}\StringTok{ "Unknown Bar"}
  \NormalTok{\}else\{}
    \NormalTok{if (}\KeywordTok{length}\NormalTok{(}\KeywordTok{grep}\NormalTok{(}\StringTok{"1000"}\NormalTok{, trials$image_name[i])) >}\StringTok{ }\DecValTok{0}\NormalTok{)}
      \NormalTok{time_chart <-}\StringTok{ 'One Second Pie'}
    \NormalTok{else if (}\KeywordTok{length}\NormalTok{(}\KeywordTok{grep}\NormalTok{(}\StringTok{"0500"}\NormalTok{, trials$image_name[i])) >}\StringTok{ }\DecValTok{0}\NormalTok{)}
      \NormalTok{time_chart <-}\StringTok{ 'Half Second Pie'}
    \NormalTok{else if (}\KeywordTok{length}\NormalTok{(}\KeywordTok{grep}\NormalTok{(}\StringTok{"3000"}\NormalTok{, trials$image_name[i])) >}\StringTok{ }\DecValTok{0}\NormalTok{)}
      \NormalTok{time_chart <-}\StringTok{ 'Three Seconds Pie'}
    \NormalTok{else}
      \NormalTok{time_chart <-}\StringTok{ "Unknown Pie"}
  \NormalTok{\}}
  \NormalTok{trials$time_chart[i] <-}\StringTok{ }\NormalTok{time_chart}
\NormalTok{\}}
\end{Highlighting}
\end{Shaded}

\subsubsection{Create column for the log\_2 error rate (as described in
the paper by Cleveland and
McGill)}\label{create-column-for-the-log_2-error-rate-as-described-in-the-paper-by-cleveland-and-mcgill}

\begin{Shaded}
\begin{Highlighting}[]
\NormalTok{trials$log2_error <-}\StringTok{ }\KeywordTok{log}\NormalTok{(}\KeywordTok{abs}\NormalTok{(}\KeywordTok{strtoi}\NormalTok{(trials$actual_answer) -}\StringTok{ }\KeywordTok{strtoi}\NormalTok{(trials$expected_answer)) +}\StringTok{ }\FloatTok{0.125}\NormalTok{, }\DataTypeTok{base =} \DecValTok{2}\NormalTok{)}
\end{Highlighting}
\end{Shaded}

\subsubsection{\texorpdfstring{Create \texttt{participant} column from
\texttt{session\_id} (to make the facet plot
tidier)}{Create participant column from session\_id (to make the facet plot tidier)}}\label{create-participant-column-from-session_id-to-make-the-facet-plot-tidier}

\begin{Shaded}
\begin{Highlighting}[]
\NormalTok{trials$participant <-}\StringTok{ }\KeywordTok{factor}\NormalTok{(}
  \NormalTok{trials$session_id, }\DataTypeTok{levels=}\KeywordTok{unique}\NormalTok{(trials$session_id), }\DataTypeTok{labels =} \KeywordTok{seq_along}\NormalTok{(}\KeywordTok{unique}\NormalTok{(trials$session_id))}
\NormalTok{)}

\NormalTok{trials %>%}\StringTok{ }\KeywordTok{filter}\NormalTok{(participant ==}\StringTok{ }\DecValTok{45}\NormalTok{)}
\end{Highlighting}
\end{Shaded}

\begin{verbatim}
## # A tibble: 0 x 9
## # ... with 9 variables: actual_answer <chr>, expected_answer <chr>,
## #   image_name <chr>, session_id <chr>, condition <chr>, time <chr>,
## #   time_chart <chr>, log2_error <dbl>, participant <fct>
\end{verbatim}

\section{Comparing aggregate error rates for bar and pie
charts}\label{comparing-aggregate-error-rates-for-bar-and-pie-charts}

\begin{Shaded}
\begin{Highlighting}[]
\NormalTok{trials %>%}
\StringTok{  }\KeywordTok{ggplot}\NormalTok{(}\KeywordTok{aes}\NormalTok{(}\DataTypeTok{x =} \NormalTok{condition, }\DataTypeTok{y =} \NormalTok{log2_error)) +}
\StringTok{  }\KeywordTok{geom_point}\NormalTok{(}\DataTypeTok{alpha =} \FloatTok{0.5}\NormalTok{) +}
\StringTok{  }\KeywordTok{stat_summary}\NormalTok{(}\DataTypeTok{fun.data =} \StringTok{"mean_cl_boot"}\NormalTok{, }\DataTypeTok{colour =} \StringTok{"red"}\NormalTok{, }\DataTypeTok{size =} \NormalTok{.}\DecValTok{50}\NormalTok{, }\DataTypeTok{alpha=}\DecValTok{1}\NormalTok{) +}
\StringTok{  }\KeywordTok{coord_flip}\NormalTok{() +}
\StringTok{  }\KeywordTok{theme}\NormalTok{(}\DataTypeTok{plot.title =} \KeywordTok{element_text}\NormalTok{(}\DataTypeTok{hjust =} \FloatTok{0.5}\NormalTok{)) +}
\StringTok{  }\KeywordTok{ggtitle}\NormalTok{(}\StringTok{"Aggregated error rates by chart type"}\NormalTok{)}
\end{Highlighting}
\end{Shaded}

\begin{verbatim}
## Warning: Removed 12 rows containing non-finite values (stat_summary).
\end{verbatim}

\begin{verbatim}
## Warning: Removed 12 rows containing missing values (geom_point).
\end{verbatim}

\includegraphics{analysis_files/figure-latex/unnamed-chunk-10-1.pdf}

\begin{Shaded}
\begin{Highlighting}[]
\NormalTok{trials %>%}
\StringTok{  }\KeywordTok{ggplot}\NormalTok{(}\KeywordTok{aes}\NormalTok{(}\DataTypeTok{x =} \NormalTok{time, }\DataTypeTok{y =} \NormalTok{log2_error)) +}
\StringTok{  }\KeywordTok{geom_point}\NormalTok{(}\DataTypeTok{alpha =} \FloatTok{0.5}\NormalTok{) +}
\StringTok{  }\KeywordTok{stat_summary}\NormalTok{(}\DataTypeTok{fun.data =} \StringTok{"mean_cl_boot"}\NormalTok{, }\DataTypeTok{colour =} \StringTok{"red"}\NormalTok{, }\DataTypeTok{size =} \FloatTok{1.0}\NormalTok{, }\DataTypeTok{alpha=}\FloatTok{0.5}\NormalTok{) +}
\StringTok{  }\KeywordTok{coord_flip}\NormalTok{() +}
\StringTok{  }\KeywordTok{theme}\NormalTok{(}\DataTypeTok{plot.title =} \KeywordTok{element_text}\NormalTok{(}\DataTypeTok{hjust =} \FloatTok{0.5}\NormalTok{)) +}
\StringTok{  }\KeywordTok{ggtitle}\NormalTok{(}\StringTok{"Aggregated error rates by chart type"}\NormalTok{)}
\end{Highlighting}
\end{Shaded}

\begin{verbatim}
## Warning: Removed 12 rows containing non-finite values (stat_summary).
\end{verbatim}

\begin{verbatim}
## Warning: Removed 12 rows containing missing values (geom_point).
\end{verbatim}

\includegraphics{analysis_files/figure-latex/unnamed-chunk-11-1.pdf}

\begin{Shaded}
\begin{Highlighting}[]
\NormalTok{trials %>%}
\StringTok{  }\KeywordTok{ggplot}\NormalTok{(}\KeywordTok{aes}\NormalTok{(}\DataTypeTok{x =} \NormalTok{time_chart, }\DataTypeTok{y =} \NormalTok{log2_error)) +}
\StringTok{  }\KeywordTok{geom_point}\NormalTok{(}\DataTypeTok{alpha =} \FloatTok{0.5}\NormalTok{) +}
\StringTok{  }\KeywordTok{stat_summary}\NormalTok{(}\DataTypeTok{fun.data =} \StringTok{"mean_cl_boot"}\NormalTok{, }\DataTypeTok{colour =} \StringTok{"red"}\NormalTok{, }\DataTypeTok{size =} \FloatTok{1.0}\NormalTok{, }\DataTypeTok{alpha=}\FloatTok{0.5}\NormalTok{) +}
\StringTok{  }\KeywordTok{coord_flip}\NormalTok{() +}
\StringTok{  }\KeywordTok{theme}\NormalTok{(}\DataTypeTok{plot.title =} \KeywordTok{element_text}\NormalTok{(}\DataTypeTok{hjust =} \FloatTok{0.5}\NormalTok{)) +}
\StringTok{  }\KeywordTok{ggtitle}\NormalTok{(}\StringTok{"Aggregated error rates by chart type"}\NormalTok{)}
\end{Highlighting}
\end{Shaded}

\begin{verbatim}
## Warning: Removed 12 rows containing non-finite values (stat_summary).
\end{verbatim}

\begin{verbatim}
## Warning: Removed 12 rows containing missing values (geom_point).
\end{verbatim}

\includegraphics{analysis_files/figure-latex/unnamed-chunk-12-1.pdf} As
expected, error rates on pie charts are higher than the error rates on
bar charts

\section{Comparing individual participant error rates by chart
type}\label{comparing-individual-participant-error-rates-by-chart-type}

\begin{Shaded}
\begin{Highlighting}[]
\NormalTok{trials %>%}
\StringTok{  }\KeywordTok{ggplot}\NormalTok{(}\KeywordTok{aes}\NormalTok{(}\DataTypeTok{x =} \NormalTok{time, }\DataTypeTok{y =} \NormalTok{log2_error)) +}
\StringTok{  }\KeywordTok{geom_point}\NormalTok{(}\DataTypeTok{alpha =} \FloatTok{0.5}\NormalTok{) +}
\StringTok{  }\KeywordTok{stat_summary}\NormalTok{(}\DataTypeTok{fun.data =} \StringTok{"mean_cl_boot"}\NormalTok{, }\DataTypeTok{colour =} \StringTok{"red"}\NormalTok{, }\DataTypeTok{size =} \NormalTok{.}\DecValTok{3}\NormalTok{, }\DataTypeTok{alpha=}\FloatTok{0.5}\NormalTok{) +}
\StringTok{  }\KeywordTok{facet_wrap}\NormalTok{(~}\StringTok{ }\NormalTok{participant) +}
\StringTok{  }\KeywordTok{theme}\NormalTok{(}\DataTypeTok{plot.title =} \KeywordTok{element_text}\NormalTok{(}\DataTypeTok{hjust =} \FloatTok{0.5}\NormalTok{)) +}
\StringTok{  }\KeywordTok{ggtitle}\NormalTok{(}\StringTok{"Individual error rates by chart type"}\NormalTok{)}
\end{Highlighting}
\end{Shaded}

\begin{verbatim}
## Warning: Removed 12 rows containing non-finite values (stat_summary).
\end{verbatim}

\begin{verbatim}
## Warning: Removed 12 rows containing missing values (geom_point).
\end{verbatim}

\includegraphics{analysis_files/figure-latex/unnamed-chunk-13-1.pdf}

\begin{Shaded}
\begin{Highlighting}[]
   \KeywordTok{theme}\NormalTok{(}\DataTypeTok{panel.spacing =} \KeywordTok{unit}\NormalTok{(}\DecValTok{2}\NormalTok{, }\StringTok{"lines"}\NormalTok{))}
\end{Highlighting}
\end{Shaded}

\begin{verbatim}
## List of 1
##  $ panel.spacing: 'unit' num 2lines
##   ..- attr(*, "valid.unit")= int 3
##   ..- attr(*, "unit")= chr "lines"
##  - attr(*, "class")= chr [1:2] "theme" "gg"
##  - attr(*, "complete")= logi FALSE
##  - attr(*, "validate")= logi TRUE
\end{verbatim}

\begin{Shaded}
\begin{Highlighting}[]
\KeywordTok{ggsave}\NormalTok{(}\StringTok{'barvpie_byparticipant.pdf'}\NormalTok{, }\DataTypeTok{units=}\StringTok{"in"}\NormalTok{, }\DataTypeTok{width=}\DecValTok{8}\NormalTok{, }\DataTypeTok{height=}\DecValTok{11}\NormalTok{)}
\end{Highlighting}
\end{Shaded}

\begin{verbatim}
## Warning: Removed 12 rows containing non-finite values (stat_summary).

## Warning: Removed 12 rows containing missing values (geom_point).
\end{verbatim}

We still see the same trend of higher error rates on pie charts (with a
couple of outliers)

\section{Statistical Analysis}\label{statistical-analysis}

\subsection{Normality test of the bar and pie
responses}\label{normality-test-of-the-bar-and-pie-responses}

\begin{Shaded}
\begin{Highlighting}[]
\NormalTok{pie <-}\StringTok{ }\NormalTok{trials %>%}\StringTok{ }
\StringTok{  }\KeywordTok{filter}\NormalTok{(condition ==}\StringTok{ "Pie"}\NormalTok{)}
\NormalTok{bar <-}\StringTok{ }\NormalTok{trials %>%}
\StringTok{  }\KeywordTok{filter}\NormalTok{(condition ==}\StringTok{ "Bar"}\NormalTok{)}

\NormalTok{bar_time_half <-}\StringTok{ }\NormalTok{trials %>%}
\StringTok{  }\KeywordTok{filter}\NormalTok{(time_chart ==}\StringTok{ "Half Second Bar"}\NormalTok{)}
\NormalTok{bar_time_one <-}\StringTok{ }\NormalTok{trials %>%}
\StringTok{  }\KeywordTok{filter}\NormalTok{(time_chart ==}\StringTok{ "One Second Bar"}\NormalTok{)}
\NormalTok{bar_time_three <-}\StringTok{ }\NormalTok{trials %>%}
\StringTok{  }\KeywordTok{filter}\NormalTok{(time_chart ==}\StringTok{ "Three Seconds Bar"}\NormalTok{)}

\NormalTok{pie_time_half <-}\StringTok{ }\NormalTok{trials %>%}
\StringTok{  }\KeywordTok{filter}\NormalTok{(time_chart ==}\StringTok{ 'Half Second Pie'}\NormalTok{)}
\NormalTok{pie_time_one <-}\StringTok{ }\NormalTok{trials %>%}
\StringTok{  }\KeywordTok{filter}\NormalTok{(time_chart ==}\StringTok{ "One Second Pie"}\NormalTok{)}
\NormalTok{pie_time_three <-}\StringTok{ }\NormalTok{trials %>%}
\StringTok{  }\KeywordTok{filter}\NormalTok{(time_chart ==}\StringTok{ "Three Seconds Pie"}\NormalTok{)}
\end{Highlighting}
\end{Shaded}

\begin{Shaded}
\begin{Highlighting}[]
\KeywordTok{shapiro.test}\NormalTok{(bar$log2_error)}
\end{Highlighting}
\end{Shaded}

\begin{verbatim}
## 
##  Shapiro-Wilk normality test
## 
## data:  bar$log2_error
## W = 0.78159, p-value < 2.2e-16
\end{verbatim}

\begin{Shaded}
\begin{Highlighting}[]
\KeywordTok{shapiro.test}\NormalTok{(pie$log2_error)}
\end{Highlighting}
\end{Shaded}

\begin{verbatim}
## 
##  Shapiro-Wilk normality test
## 
## data:  pie$log2_error
## W = 0.78531, p-value < 2.2e-16
\end{verbatim}

\begin{Shaded}
\begin{Highlighting}[]
\KeywordTok{shapiro.test}\NormalTok{(pie_time_half$log2_error)}
\end{Highlighting}
\end{Shaded}

\begin{verbatim}
## 
##  Shapiro-Wilk normality test
## 
## data:  pie_time_half$log2_error
## W = 0.76007, p-value < 2.2e-16
\end{verbatim}

\begin{Shaded}
\begin{Highlighting}[]
\KeywordTok{shapiro.test}\NormalTok{(pie_time_one$log2_error)}
\end{Highlighting}
\end{Shaded}

\begin{verbatim}
## 
##  Shapiro-Wilk normality test
## 
## data:  pie_time_one$log2_error
## W = 0.81257, p-value = 6.234e-15
\end{verbatim}

\begin{Shaded}
\begin{Highlighting}[]
\KeywordTok{shapiro.test}\NormalTok{(pie_time_three$log2_error)}
\end{Highlighting}
\end{Shaded}

\begin{verbatim}
## 
##  Shapiro-Wilk normality test
## 
## data:  pie_time_three$log2_error
## W = 0.74821, p-value < 2.2e-16
\end{verbatim}

\begin{Shaded}
\begin{Highlighting}[]
\KeywordTok{shapiro.test}\NormalTok{(bar_time_half$log2_error)}
\end{Highlighting}
\end{Shaded}

\begin{verbatim}
## 
##  Shapiro-Wilk normality test
## 
## data:  bar_time_half$log2_error
## W = 0.76811, p-value < 2.2e-16
\end{verbatim}

\begin{Shaded}
\begin{Highlighting}[]
\KeywordTok{shapiro.test}\NormalTok{(bar_time_one$log2_error)}
\end{Highlighting}
\end{Shaded}

\begin{verbatim}
## 
##  Shapiro-Wilk normality test
## 
## data:  bar_time_one$log2_error
## W = 0.80134, p-value = 2.61e-15
\end{verbatim}

\begin{Shaded}
\begin{Highlighting}[]
\KeywordTok{shapiro.test}\NormalTok{(bar_time_three$log2_error)}
\end{Highlighting}
\end{Shaded}

\begin{verbatim}
## 
##  Shapiro-Wilk normality test
## 
## data:  bar_time_three$log2_error
## W = 0.70937, p-value < 2.2e-16
\end{verbatim}

\subsection{Density plots - validating normality
test}\label{density-plots---validating-normality-test}

Our normality test indicates that the error rates for the bar and pie
charts are not normally distributed. We validate this by creating
density plots by chart type, with the mean value of each group also
indicated.

\begin{Shaded}
\begin{Highlighting}[]
\NormalTok{mu <-}\StringTok{ }\KeywordTok{ddply}\NormalTok{(trials, }\StringTok{"condition"}\NormalTok{, summarise, }\DataTypeTok{grp.mean=}\KeywordTok{mean}\NormalTok{(log2_error))}

\NormalTok{trials %>%}
\StringTok{  }\KeywordTok{ggplot}\NormalTok{(}\KeywordTok{aes}\NormalTok{(}\DataTypeTok{x=}\NormalTok{log2_error, }\DataTypeTok{color=}\NormalTok{condition)) +}
\StringTok{  }\KeywordTok{geom_density}\NormalTok{() +}
\StringTok{  }\KeywordTok{geom_vline}\NormalTok{(}\DataTypeTok{data=}\NormalTok{mu, }\KeywordTok{aes}\NormalTok{(}\DataTypeTok{xintercept=}\NormalTok{grp.mean, }\DataTypeTok{color=}\NormalTok{condition),}
             \DataTypeTok{linetype=}\StringTok{"dashed"}\NormalTok{)}
\end{Highlighting}
\end{Shaded}

\begin{verbatim}
## Warning: Removed 12 rows containing non-finite values (stat_density).
\end{verbatim}

\begin{verbatim}
## Warning: Removed 2 rows containing missing values (geom_vline).
\end{verbatim}

\includegraphics{analysis_files/figure-latex/unnamed-chunk-16-1.pdf}

\begin{Shaded}
\begin{Highlighting}[]
\NormalTok{mu2 <-}\StringTok{ }\KeywordTok{ddply}\NormalTok{(trials, }\StringTok{"time"}\NormalTok{, summarise, }\DataTypeTok{grp.mean=}\KeywordTok{mean}\NormalTok{(log2_error))}

\NormalTok{trials %>%}
\StringTok{  }\KeywordTok{ggplot}\NormalTok{(}\KeywordTok{aes}\NormalTok{(}\DataTypeTok{x=}\NormalTok{log2_error, }\DataTypeTok{color=}\NormalTok{time)) +}
\StringTok{  }\KeywordTok{geom_density}\NormalTok{() +}
\StringTok{  }\KeywordTok{geom_vline}\NormalTok{(}\DataTypeTok{data=}\NormalTok{mu2, }\KeywordTok{aes}\NormalTok{(}\DataTypeTok{xintercept=}\NormalTok{grp.mean, }\DataTypeTok{color=}\NormalTok{time),}
             \DataTypeTok{linetype=}\StringTok{"dashed"}\NormalTok{)}
\end{Highlighting}
\end{Shaded}

\begin{verbatim}
## Warning: Removed 12 rows containing non-finite values (stat_density).
\end{verbatim}

\begin{verbatim}
## Warning: Removed 3 rows containing missing values (geom_vline).
\end{verbatim}

\includegraphics{analysis_files/figure-latex/unnamed-chunk-17-1.pdf}

\subsection{Wilcoxon rank sum test}\label{wilcoxon-rank-sum-test}

Our test for normality came out negative therefore we perform the
Wilcoxon rank sum test.

\begin{Shaded}
\begin{Highlighting}[]
\KeywordTok{wilcox.test}\NormalTok{(bar_time_half$log2_error, bar_time_one$log2_error)}
\end{Highlighting}
\end{Shaded}

\begin{verbatim}
## 
##  Wilcoxon rank sum test with continuity correction
## 
## data:  bar_time_half$log2_error and bar_time_one$log2_error
## W = 21352, p-value = 0.4707
## alternative hypothesis: true location shift is not equal to 0
\end{verbatim}

\begin{Shaded}
\begin{Highlighting}[]
\KeywordTok{wilcox.test}\NormalTok{(bar_time_half$log2_error, bar_time_three$log2_error)}
\end{Highlighting}
\end{Shaded}

\begin{verbatim}
## 
##  Wilcoxon rank sum test with continuity correction
## 
## data:  bar_time_half$log2_error and bar_time_three$log2_error
## W = 23819, p-value = 0.002567
## alternative hypothesis: true location shift is not equal to 0
\end{verbatim}

\begin{Shaded}
\begin{Highlighting}[]
\KeywordTok{wilcox.test}\NormalTok{(bar_time_one$log2_error, bar_time_three$log2_error)}
\end{Highlighting}
\end{Shaded}

\begin{verbatim}
## 
##  Wilcoxon rank sum test with continuity correction
## 
## data:  bar_time_one$log2_error and bar_time_three$log2_error
## W = 22484, p-value = 0.04951
## alternative hypothesis: true location shift is not equal to 0
\end{verbatim}

\begin{Shaded}
\begin{Highlighting}[]
\KeywordTok{wilcox.test}\NormalTok{(pie_time_half$log2_error, pie_time_one$log2_error)}
\end{Highlighting}
\end{Shaded}

\begin{verbatim}
## 
##  Wilcoxon rank sum test with continuity correction
## 
## data:  pie_time_half$log2_error and pie_time_one$log2_error
## W = 22298, p-value = 0.2833
## alternative hypothesis: true location shift is not equal to 0
\end{verbatim}

\begin{Shaded}
\begin{Highlighting}[]
\KeywordTok{wilcox.test}\NormalTok{(pie_time_half$log2_error, pie_time_three$log2_error)}
\end{Highlighting}
\end{Shaded}

\begin{verbatim}
## 
##  Wilcoxon rank sum test with continuity correction
## 
## data:  pie_time_half$log2_error and pie_time_three$log2_error
## W = 23886, p-value = 0.02119
## alternative hypothesis: true location shift is not equal to 0
\end{verbatim}

\begin{Shaded}
\begin{Highlighting}[]
\KeywordTok{wilcox.test}\NormalTok{(pie_time_one$log2_error, pie_time_three$log2_error)}
\end{Highlighting}
\end{Shaded}

\begin{verbatim}
## 
##  Wilcoxon rank sum test with continuity correction
## 
## data:  pie_time_one$log2_error and pie_time_three$log2_error
## W = 22373, p-value = 0.2201
## alternative hypothesis: true location shift is not equal to 0
\end{verbatim}

\begin{Shaded}
\begin{Highlighting}[]
\KeywordTok{wilcox.test}\NormalTok{(pie_time_half$log2_error, bar_time_half$log2_error)}
\end{Highlighting}
\end{Shaded}

\begin{verbatim}
## 
##  Wilcoxon rank sum test with continuity correction
## 
## data:  pie_time_half$log2_error and bar_time_half$log2_error
## W = 19896, p-value = 0.3963
## alternative hypothesis: true location shift is not equal to 0
\end{verbatim}


\end{document}
